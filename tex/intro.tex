% ==============================================================================
% LAB 119
% UNDERSÖKNING AV RC-KRETS
% ------------------------
%
% Author:
% Jonas Sjöberg     <tel12jsg@student.hig.se>
% Oscar Wallberg    <tco13owg@student.hig.se>
%
% License:
% Creative Commons Attribution-NonCommercial-ShareAlike 4.0 International
% See LICENSE.md for full licensing information.
% ==============================================================================

\section{Introduktion}\label{intro}
I denna labb skall vi studera en passiv krets uppbyggd av ett motstånd och en
kondensator. Om en sådan krets matas med en sinusformad insignal kommer den att
släppa igenom vissa frekvenser medan andra frekvenser dämpas. Ett sådant
frekvensberoende nät kallas därför ofta för filter. Om kretsen innehåller
endast en reaktiv (dvs energilagrande) komponent (spole eller kondensator)
kallar vi kretsen för ett första ordningens filter. Namnet kommer sig av att
kretsen kan beskrivas med en första ordningens differentialekvation.  Vi skall
analysera kretsen både i frekvensplanet genom att mäta upp ett Bode-diagram och
i tidsplanet genom att mäta upp kretsens stegsvar.  Ett första ordningens
lågpassfilter kan konstrueras enligt Figur~\ref{fig:rc-schema}.

\section{Lågpassfilter}
\subsection{Överföringsfunktion}
Uttryck \eqref{eq:transfer} beskriver lågpassfiltrets överföringsfunktion i
Bodes normalform.

\begin{equation*}
  \begin{split}
    H(j\omega) &= \dfrac{U_{ut}}{U_{in}}                                      \\
               &= \dfrac{\dfrac{1}{\jmath\omega C}}{R + \dfrac{1}{j\omega C}} \\
               &= \dfrac{1}{1+\jmath\omega R C}                               \\
  \end{split}
\end{equation*}

\begin{equation}\label{eq:transfer}
  H(j\omega) = \dfrac{1}{1+j(\omega/\omega_1)}
\end{equation}

där $\omega_1 = \tfrac{1}{R C}$ är brytfrekvensen uttryckt som en
vinkelfrekvens $\si{\radian\per\second}$.


\par 
Eftersom $\omega = 2 \pi f$ kan vi också uttrycka överföringsfunktionen som:

\begin{equation*}
  \begin{split}
    H(f) &= \dfrac{U_{ut}}{U_{in}}        \\
    H(f) &= \dfrac{1}{1+\jmath 2 \pi R C} \\
  \end{split}
\end{equation*}

\begin{equation}\label{eq:transfer2}
  H(f) = \dfrac{1}{1+\jmath (\dfrac{f}{f_1})}\
  \text{där}\ f_1 = \dfrac{1}{2 \pi R C} \si{\Hz}
\end{equation}


\par Den senare formen \eqref{eq:transfer2} är att föredra när man plottar
upp överföringsfunktionen från mätresultatet och är den form vi använder i
labben.

Eftersom överföringsfunktionen är på komplex form har den både absolutbelopp
och fasvinkel \eqref{eq:complexform}:

\begin{equation}\label{eq:complexform}
  \begin{split}
    |H(f)| &= \dfrac{1}{\sqrt{1+(\frac{f}{f_1})^2}} \\
      ArgH &= -\arctan{\frac{f}{f_1}}
  \end{split}
\end{equation}


%$z = 1.19 \phase{-78.2039^{\circ}}$


