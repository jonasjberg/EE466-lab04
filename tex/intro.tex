% ==============================================================================
% LAB 119
% UNDERSÖKNING AV RC-KRETS
% ------------------------
%
% Author:
% Jonas Sjöberg     <tel12jsg@student.hig.se>
% Oscar Wallberg    <tco13owg@student.hig.se>
%
% License:
% Creative Commons Attribution-NonCommercial-ShareAlike 4.0 International
% See LICENSE.md for full licensing information.
% ==============================================================================

\section{Introduktion}\label{intro}
I denna labb skall vi studera en passiv krets uppbyggd av ett motstånd och en
kondensator. Om en sådan krets matas med en sinusformad insignal kommer den att
släppa igenom vissa frekvenser medan andra frekvenser dämpas. Ett sådant
frekvensberoende nät kallas därför ofta för filter. Om kretsen innehåller
endast en reaktiv (dvs energilagrande) komponent (spole eller kondensator)
kallar vi kretsen för ett första ordningens filter. Namnet kommer sig av att
kretsen kan beskrivas med en första ordningens differentialekvation.
Vi skall analysera kretsen både i frekvensplanet genom att mäta upp ett
Bode-diagram och i tidsplanet genom att mäta upp kretsens stegsvar.

