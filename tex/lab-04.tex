% ==============================================================================
% LAB 119
% UNDERSÖKNING AV RC-KRETS
% ------------------------
%
% Author:
% Jonas Sjöberg     <tel12jsg@student.hig.se>
% Oscar Wallberg    <tco13owg@student.hig.se>
%
% License:
% Creative Commons Attribution-NonCommercial-ShareAlike 4.0 International
% See LICENSE.md for full licensing information.
% ==============================================================================

% ==============================================================================
% INCLUDES AND CONFIGURATION
% ==============================================================================
\documentclass[11pt,a4paper]{article}
\usepackage[utf8]{inputenc}
\usepackage[swedish]{babel} % För svensk innehållsförteckning
\usepackage{siunitx} % (För dokumentation, kör i terminalen; texdoc siunitx)
\usepackage{amssymb}
\usepackage{amsmath}
\usepackage{amsfonts}
\usepackage{graphicx}
\usepackage{booktabs}
\usepackage{longtable} % Tables span across pages
\usepackage{microtype}
\usepackage{gensymb}
%\usepackage{tabto}
\usepackage{units}

\setlength\parindent{0pt} % Removes all indentation from paragraphs

% ==============================================================================
% DOCUMENT METADATA
% ==============================================================================
\title{EE466 \\ Lab 119 \\ Undersökning av RC-krets}

\author{                                 \\
  Jonas Sjöberg                          \\
  860224                                 \\
  Högskolan i Gävle,                     \\
  Elektronikingenjörsprogrammet,         \\
  \texttt{tel12jsg@student.hig.se}       \\
  \texttt{https://github.com/jonasjberg} \\
                                         \\
  Oscar Wallberg                         \\
  Högskolan i Gävle,                     \\
  Dataingenjörsprogrammet,               \\
  \texttt{tco13owg@student.hig.se}       \\
}

\date{}

\begin{document}
\maketitle

\begin{center}
    \begin{tabular}{l r}
        Labb utförd: & TODO: Labben utförd datum \\
        Instruktör: & Efrain Zenteno
    \end{tabular}
\end{center}

\medskip

\begin{abstract}
    \begin{center}
        Laborationsrapport för \emph{EE466 -- Elektrisk kretsteori}, Högskolan
        i Gävle.  \par Syftet med laborationen är att analysera funktionen hos
        en RC krets.  Laborationen innefattar överföringsfunktionen för en
        RC-krets, i både tids- och frekvensdomänen. Stegsvaret för en första
        ordningens krets.  Bode-diagram.  Begreppen brytfrekvens, frekvens- och
        faskaraktäristik.
    \end{center}
\end{abstract}

\newpage

{
    %\hypersetup{linkcolor=black}
    \setcounter{tocdepth}{3}
    \tableofcontents
}

\newpage

\input intro.tex
\input bode.tex
\input stegsvar.tex
\input impedans.tex


\section{Inverkan av källimpedans och belastningsimpedans}\label{}
% ==============================================================================
% TODO: 

\subsection{Mätresultat}\label{}
% ------------------------------------------------------------------------------
% TODO:

\subsection{Kommentar}\label{}
% ------------------------------------------------------------------------------
% TODO:


% ==============================================================================
% SECTION: RESULTAT
% ==============================================================================
\section{Resultat}\label{setup}
% ==============================================================================
Sammanfattningsvis kan sägas att laborationen innehåller en mängd koncept som
är mycket viktiga att få en grundlig förståelse för. Vi har inte stött på några
direkta problem.

\newpage

% ==============================================================================
% SECTION: REFERENSER
% ==============================================================================
\section{Referenser}\label{refs}
% ==============================================================================
%TODO: Referenser.

%\subsection{www}\label{interwebs}
% ------------------------------------------------------------------------------

%\subsection{Trycksaker}\label{literature} %???
% ------------------------------------------------------------------------------

%\subsection{Källkod}\label{sourcefiles}
% ------------------------------------------------------------------------------

% ==============================================================================
\end{document}
% ==============================================================================
